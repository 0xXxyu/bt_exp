% !TEX root = btstack_gettingstarted.tex
\section{L2CAP API}
\label{appendix:api_l2cap}
$ $
\begin{lstlisting}
// Set up L2CAP and register L2CAP with HCI layer.
void l2cap_init(void);

// Registers a packet handler that handles HCI and general BTstack
// events.
void l2cap_register_packet_handler(void (*handler)(void * connection, uint8_t packet_type, uint16_t channel, uint8_t *packet, uint16_t size));

// Creates L2CAP channel to the PSM of a remote device with baseband
// address. A new baseband connection will be initiated if needed.
void l2cap_create_channel_internal(void * connection, btstack_packet_handler_t packet_handler, bd_addr_t address, uint16_t psm, uint16_t mtu);

// Disconencts L2CAP channel with given identifier. 
void l2cap_disconnect_internal(uint16_t local_cid, uint8_t reason);

// Queries the maximal transfer unit (MTU) for L2CAP channel with
// given identifier. 
uint16_t l2cap_get_remote_mtu_for_local_cid(uint16_t local_cid);

// Sends L2CAP data packet to the channel with given identifier.
int l2cap_send_internal(uint16_t local_cid, uint8_t *data, uint16_t len);

// Registers L2CAP service with given PSM and MTU, and assigns a
// packet handler. On embedded systems, use NULL for connection
// parameter.
void l2cap_register_service_internal(void *connection, btstack_packet_handler_t packet_handler, uint16_t psm, uint16_t mtu);

// Unregisters L2CAP service with given PSM.  On embedded systems,
// use NULL for connection parameter.
void l2cap_unregister_service_internal(void *connection, uint16_t psm);

// Accepts/Deny incoming L2CAP connection.
void l2cap_accept_connection_internal(uint16_t local_cid);
void l2cap_decline_connection_internal(uint16_t local_cid, uint8_t reason);
\end{lstlisting}
\pagebreak